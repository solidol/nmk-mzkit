\chapter{Матеріали до Л/Р \No 5}



\section{Функції для роботи з файловою системою}
\index{Файл!функції}
\begin{longtable}[t]{|l|p{20em}|}
\kill
\caption{\space Перелік функцій для роботи з ФС} \label{fs-funcr:table}\\
\hline

функція & Опис \\
\hline \endfirsthead
\caption*{\space Продовження} \\
\hline
функція & Опис \\
\hline \endhead
\hline \endfoot



\verb'basename' & Повертає останній компонент імені з вказаного шляху\\
\verb'chgrp' & Змінює групу власників файлу\\
\verb'chmod' & Змінює режим доступу до файлу\\
\verb'chown' & Змінює власника файлу\\
\verb'clearstatcache' & Очищає кеш стану файлів\\
\verb'copy' & Копіює файл\\
\verb'delete' & См.опис функції unlink або unset\\
\verb'dirname' & Повертає ім'я батьківського каталогу з зазначеного шляху\\
\verb'disk_free_space' & Повертає розмір доступного простору в каталозі або в файловій системі\\
\verb'disk_total_space' & Повертає загальний розмір каталогу або розділу файлової системи\\
\verb'diskfreespace' & Псевдонім disk\_free\_space\\
\verb'fclose' & Закриває відкритий дескриптор файлу\\
\verb'feof' & Перевіряє, чи досягнутий кінець файлу\\
\verb'fflush' & Скидає буфер виводу у файл\\
\verb'fgetc' & Зчитує символ з файлу\\
\verb'fgetcsv' & Читає рядок з файлу і виробляє розбір даних CSV\\
\verb'fgets' & Читає рядок з файлу\\
\verb'fgetss' & Читає рядок з файлу і відкидає HTML-теги\\
\verb'file_exists' & Перевіряє наявність вказаного файлу або каталогу\\
\verb'file_get_contents' & Читає вміст файлу в рядок\\
\verb'file_put_contents' & Пише рядок в файл\\
\verb'file' & Читає вміст файлу і поміщає його в масив\\
\verb'fileatime' & Повертає час останнього доступу до файлу\\
\verb'filectime' & Повертає час зміни індексного дескриптора файлу\\
\verb'filegroup' & Отримує ідентифікатор групи файлу\\
\verb'fileinode' & Повертає індексний дескриптор файлу\\
\verb'filemtime' & Повертає час останньої зміни файлу\\
\verb'fileowner' & Повертає ідентифікатор власника файлу\\
\verb'fileperms' & Повертає інформацію про права на файл\\
\verb'filesize' & Повертає розмір файлу\\
\verb'filetype' & Повертає тип файлу\\
\verb'flock' & Переносима консультативна блокування файлів\\
\verb'fnmatch' & Перевіряє збіг імені файлу з шаблоном\\
\verb'fopen' & Відкриває файл або URL\\
\verb'fpassthru' & Виводить всі залишилися дані з файлового покажчика\\
\verb'fputcsv' & Форматує рядок у вигляді CSV і записує його в файловий покажчик\\
\verb'fputs' & Псевдонім fwrite\\
\verb'fread' & бінарних-безпечне читання файлу\\
\verb'fscanf' & Обробляє дані з файлу відповідно до формату\\
\verb'fseek' & Встановлює зсув у файловому покажчику\\
\verb'fstat' & Отримує інформацію про фото використовуючи відкритий файловий покажчик\\
\verb'ftell' & Повідомляє поточну позицію читання/запису файлу\\
\verb'ftruncate' & Врізає файл до вказаної довжини\\
\verb'fwrite' & бінарно-безпечний запис в файл\\
\verb'glob' & Знаходить файлові шляху, що збігаються з шаблоном\\
\verb'is_dir' & Визначає, чи є ім'я файлу директорією\\
\verb'is_executable' & Визначає, чи є файл виконуваним\\
\verb'is_file' & Визначає, чи є файл звичайним файлом\\
\verb'is_link' & Визначає, чи є файл символічним посиланням\\
\verb'is_readable' & Визначає наявність файлу і доступний він для читання\\
\verb'is_uploaded_file' & Визначає, чи був файл завантажений за допомогою HTTP POST\\
\verb'is_writable' & Визначає, чи доступний файл для запису\\
\verb'is_writeable' & Псевдонім is\_writable\\
\verb'lchgrp' & Змінює групу, якій належить символічне посилання\\
\verb'lchown' & Змінює власника символічного посилання\\
\verb'link' & Створює жорстке посилання\\
\verb'linkinfo' & Повертає інформацію про посиланню\\
\verb'lstat' & Повертає інформацію про фото або символічної посиланню\\
\verb'mkdir' & Створює директорію\\
\verb'move_uploaded_file' & Переміщає завантажений файл у нове місце\\
\verb'parse_ini_file' & Обробляє конфігураційний файл\\
\verb'parse_ini_string' & Розбирає рядок конфігурації\\
\verb'pathinfo' & Повертає інформацію про шлях до файлу\\
\verb'pclose' & Закриває файловий покажчик процесу\\
\verb'popen' & Відкриває файловий покажчик процесу\\
\verb'readfile' & Виводить файл\\
\verb'readlink' & Повертає файл, на який вказує символьне посилання\\
\verb'realpath_cache_get' & Отримує записи з кешу реального шляху\\
\verb'realpath_cache_size' & Отримує розмір кеша реального шляху\\
\verb'realpath' & Повертає канонізований абсолютний шлях до файлу\\
\verb'rename' & Перейменовує файл або директорію\\
\verb'rewind' & Скидає курсор у файлового покажчика\\
\verb'rmdir' & Видаляє директорію\\
\verb'set_file_buffer' & Псевдонім stream\_set\_write\_buffer\\
\verb'stat' & Повертає інформацію про файл\\
\verb'symlink' & Створює символічну посилання\\
\verb'tempnam' & Створює файл з унікальним ім'ям\\
\verb'tmpfile' & Створює тимчасовий файл\\
\verb'touch' & Встановлює час доступу і модифікації файлу\\
\verb'umask' & Змінює поточну umask\\
\verb'unlink' & Видаляє файл \\


\hline
\end{longtable}



\section{Параметри функції <<fopen()>>}
\begin{longtable}[t]{|c|p{27em}|}
\kill
\caption{\space Другий параметр функції <<fopen()>>} \label{fo-par:table}\\
\hline

Параметр & Опис \\
\hline \endfirsthead
\caption*{\space Продовження} \\
\hline
Параметр & Опис \\
\hline \endhead
\hline \endfoot
\verb'R' & Відкриває файл тільки для читання; поміщає покажчик в початок файлу. \\ \hline
\verb'R+' & Відкриває файл для читання і запису; поміщає покажчик в початок файлу. \\ \hline
\verb'W' & Відкриває файл тільки для запису; поміщає покажчик в початок файлу і обрізає файл до нульової довжини. Якщо файл не існує~--- намагається його створити. \\ \hline
\verb'W+' & Відкриває файл для читання і запису; поміщає покажчик в початок файлу і обрізає файл до нульової довжини. Якщо файл не існує~--- намагається його створити. \\ \hline
\verb'A' & Відкриває файл тільки для запису; поміщає покажчик в кінець файлу. Якщо файл не існує~--- намагається його створити. \\ \hline
\verb'A+' & Відкриває файл для читання і запису; поміщає покажчик в кінець файлу. Якщо файл не існує~--- намагається його створити. \\ \hline
\verb'X' & Створює і відкриває тільки для запису; поміщає покажчик в початок файлу. Якщо файл вже існує, виклик \verb'fopen()' закінчиться невдачею, поверне \verb'FALSE' і видасть помилку \verb'E_WARNING'. Якщо файл не існує, спробує його створити. \\ \hline
\verb'X+' & Створює і відкриває для читання і запису інакше поверне \verb'FALSE'. \\ \hline
\verb'C' & Відкриває файл тільки для запису. Якщо файл не існує, то він створюється. Якщо ж файл існує, то він не обрізається, і виклик цієї функції не викликає помилку. Покажчик на файл буде встановлений на початок файлу.\\ \hline
\verb'C+' & Відкриває файл для читання і запису інакше функція має ту ж поведінку, що і з використанням <<\verb'c'>>. \\ \hline

\hline
\end{longtable}


\section{Функції для роботи з каталогами}
\index{Каталог!функції}
\begin{longtable}[t]{|c|p{27em}|}
\kill
\caption{\space Перелік функцій для роботи з каталогами} \label{dir-func:table}\\
\hline

Функція & Опис \\
\hline \endfirsthead
\caption*{\space Продовження} \\
\hline
Функція & Опис \\
\hline \endhead
\hline \endfoot
\verb'chdir' & Змінює каталог\\
\verb'mkdir' & Створює каталог\\
\verb'rmdir' & Видаляє каталог\\
\verb'is_dir' & Перевіряє чи є об'єкт каталогом\\
\verb'chroot' & Змінює кореневий каталог\\
\verb'closedir' & Звільняє дескриптор каталогу\\
\verb'dir' & Повертає екземпляр класу Directory\\
\verb'getcwd' & Отримує ім'я поточного робочого каталога\\
\verb'opendir' & Відкриває дескриптор каталогу\\
\verb'readdir' & Отримує елемент каталогу за його дескриптору\\
\verb'rewinddir' & Скинути дескриптор каталогу\\
\verb'scandir' & Отримує список файлів і каталогів, розташованих по зазначеному шляху \\

\hline
\end{longtable}
