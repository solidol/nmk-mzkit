\chapter{Матеріали до Л/Р \No 2}
\section{Елементи форм HTML}
\index{HTML!форми}
\subsection{INPUT і його методи}
\index{HTML!форми!input}
\label{inp-tag:app}
\subsection*{Однорядкові поля введення даних}

\begin{verbatim}
<input type=text name=им'я_параметра [value=значення] 
[size=розмір_поля] [maxlen=довжина_поля]>
\end{verbatim}

Даний тег створює поле вводу з максимально допустимою довжиною тексту <<maxlen>> і розміром в size <<знакомест>>. Якщо вказаний атрибут <<value>>, то в поле буде спочатку відображатися значення даного атрибуту. У квадратних дужках \verb'[]' позначені необов'язкові атрибути.

\subsection*{Поля вводу паролів}

\begin{verbatim}
<input type=password name=им'я_параметра [value=значення] 
[size=розмір_поля] [maxlen=довжина_поля]>
\end{verbatim}

Структура даного тегу така сама як і у <input type=text>. Різниця лише у відображенні даних, що вводить користувач.


\subsection*{Приховане текстове поле}

\begin{verbatim}
<input type=hidden name=им'я_параметра [value=значення]>
\end{verbatim}

Такі поля передають дані серверу, але не відображаються на сторінці. Значення атрибуту <<value>> встановлюється при формуванні сторінки, або JavaScript-сценарієм.

\subsection*{Незалежні перемикачі}

\begin{verbatim}
<input type=checkbox name=им'я_параметра [value=значення]
[checked]>
\end{verbatim} 

Перемикач виду <<прапорець>>. У разі встановлення прапорця при відправлені форми серверу будуть передані параметри <<им'я\_параметра=значення>>. Якщо прапорець не встановлено серверу взагалі нічого не буде відправлено. 

Перемикач за замовчуванням вимкнутий, щоб зробити його увімкнутим за замовчуванням треба встановити атрибут  <<checked>>.

Стан перемикача не залежить від стану інших перемикачів цього типу.

\subsection*{Залежні перемикачі}

Залежний перемикач, так само як і незалежний перемикач, може заходитись у двох станах, в залежності від атрибуту  <<checked>>. При цьому на формі може бути увімкнений лише один перемикач серед групи перемикачів з однаковим атрибутом <<name>>.
\begin{verbatim}
<form action="http://localhost/script.php" method="GET">
<input type=radio name=answer value=yes checked>Да
<input type=radio name=answer value=no>Нет
<input type=submit value=Отправить>
</form>
\end{verbatim}

\subsection*{Кнопки відправлення та очищення параметрів форми}

Кнопка відправки служить для передачі серверу змісту форми на сторінці. Атрибут <<value>> визначає текст, ща відображається на кнопці. Під час відправлення форми серверу будуть передані дані кнопки у вигляді <<им'я\_параметра=значення>>.
\begin{verbatim}
<input type=submit [name=им'я_параметра] value=значення>
\end{verbatim}
У разі використання кнопки із зображенням сереру передадуться координати кліку відносно зображення.
\begin{verbatim}
<input type=submit [name=им'я_параметра] src=зображення>
\end{verbatim}
Кнопка очищення форми знищує всі зміни внесені користувачем сайту у дану форму.
\begin{verbatim}
<input type=reset [name=им'я_параметра] value=значення>
\end{verbatim}

\subsection*{Поле вибору файлу}

Тег <input> також дозволяє активувати діалогове вікно вибору файлу та завантажувати його на сервер при відправленні форми.
\begin{verbatim}
<input type=file name=имя [value=имя_файла]>
\end{verbatim}


\subsection{Багаторядкове текстове поле}
\index{HTML!форми!textarea}
\label{tar-tag:app}
Синтаксис багаторядкового поля виглядає наступним чином:
\begin{verbatim}
<textarea name=имя [cols=ширина_в_символах] 
[rows=высота_в_символах] wrap=тип_переноса>
текст за замовчуванням
</textarea>
\end{verbatim}
Хоча висота і ширина поля необов'язкові параметри їх бажано вказувати. Атрибут <<wrap>> відповідає за перенос і може приймати наступні значення:
\begin{enumerate}
\item Virtual --- справа від тексту з'являється полоса прокрутки, а текст розбивається на рядки.
\item Physical --- залежить від браузера і виглядає по-різному
\item None --- текст залишається у тому вигляді, в якому користувач його ввів, з'являються горизонтальна і вертикальна полоси прокрутки.
\end{enumerate}

\subsection{Списки з вибором} \label{sel-tag:app}
\index{HTML!форми!select}
\subsection*{Списки з одиночним вибором}
\label{sel-tag:app}
Списки з одиночним вибором реалізуються за допомогою наступної конструкції:
\begin{verbatim}

<select name=day size=1>
<option value=1>Понедельник</option>
<option value=2>Вторник</option>
<option value=3 selected>Среда</option>
<option value=4>Четверг</option>
<option value=5>Пятница</option>
<option value=6>Суббота</option>
<option value=7>Воскресенье</option>
</select>

\end{verbatim}
При відправленні форми сервер отримає дані виду <<им'я\_параметра=значення>>. За замовчуванням може бути обраний пункт списку серед атрибутів якого є \verb'selected'.

\subsection*{Списки з множинним вибором}
Список з множинним вибором відрізняється лише атрибутом \verb'multiple' в середині тега \verb'<select>'.
\begin{verbatim}

<select name=day size=7 multiple>
<option value=1>Понедельник</option>
<option value=1>Вторник</option>
<option value=1>Среда</option>
<option value=1>Четверг</option>
<option value=1>Пятница</option>
<option value=1>Суббота</option>
<option value=1>Воскресенье</option>
</select>


\end{verbatim}


\section{Змінні-функції}
\index{PHP!функції!змінні-функції}
\label{var-func:app}
Приклад використання змінної-функції:
\begin{verbatim}
<?php
?php
function foo() {
    echo "In foo()<br />\n";
}
function bar($arg = '')
{
    echo "In bar(); argument was '$arg'.<br />\n";
}
// Функция-обертка для echo
function echoit($string)
{
    echo $string;
}

$func = 'foo';
$func();        // Вызывает функцию foo()
$func = 'bar';
$func('test');  // Вызывает функцию bar()
$func = 'echoit';
$func('test');  // Вызывает функцию echoit()
\end{verbatim}